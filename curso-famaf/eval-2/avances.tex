\documentclass{article}

\usepackage[T1]{fontenc}
\usepackage{fullpage}
\usepackage{xcolor}
\usepackage[parfill]{parskip}
\usepackage{bm}
\usepackage{xspace}
\usepackage{graphicx}
% \usepackage{amsmath}
% \usepackage{amssymb}
\usepackage{url}

\title{Formalización de la propiedad de Progreso\\para el \stlcw}
  % \\[1ex]{\LARGE Avances}}
% \subtitle{Blueprint}
\author{Cristian Sottile}
\date{20 de octubre de 2023}

\newcommand{\stlcw}{$\lambda-$cálculo simplemente tipado\xspace}
\newcommand{\stlc}{$\lambda^{\rightarrow}$}
\newcommand{\lamg}{\ensuremath{\lambda^{\mathbb{m}}}\xspace}
\newcommand{\lamm}{\ensuremath{\lambda^{\mathbb{m}}}\xspace}
\newcommand{\wme}{\ensuremath{\mathcal{W}}\xspace}
\newcommand{\wmem}[1]{\ensuremath{\mathcal{W}(#1)}\xspace}
\newcommand{\tmme}{\ensuremath{\mathcal{T}^{\mathbb{m}}}\xspace}
\newcommand{\tmmem}[1][M]{\ensuremath{\mathcal{T}^{\mathbb{m}}(#1)}\xspace}
\newcommand{\tme}{\ensuremath{\mathcal{T}}\xspace}
\newcommand{\tmem}[1][M]{\ensuremath{\mathcal{T}(#1)}\xspace}
\newcommand{\tom}{\ensuremath{\rightarrow_m}\xspace}
\newcommand{\tob}{\ensuremath{\rightarrow_{\beta}}\xspace}
\newcommand{\tof}{\ensuremath{\triangleright}\xspace}
\newcommand{\wrap}[1]{\ensuremath{\bm{\{}#1\bm{\}}}\xspace}
\newcommand{\wei}[1]{\ensuremath{\mathsf{w}(#1)}\xspace}
\newcommand{\maxdeg}[1]{\ensuremath{\dh(#1)}\xspace}
\newcommand{\simp}[1]{\ensuremath{\mathsf{S}_*(#1)}\xspace}
\newcommand{\simpd}[2][d]{\ensuremath{\mathsf{S}_{#1}(#2)}\xspace}

\newcommand{\inte}[1]{\ensuremath{[[#1]]}}
\newcommand{\lam}[2][x]{\ensuremath{\lambda #1 . #2}}

\newcommand{\sep}{\ensuremath{\ |\ }}
\newcommand{\ie}{{\em i.e.}\xspace}
\newcommand{\eg}{{\em e.g.}\xspace}
\newcommand{\ver}[1]{\textcolor{red}{#1}}

\newcommand{\n}[1]{\ensuremath{\mathsf{#1}}}

\begin{document}

\maketitle

% \begin{abstract}
%   La propuesta es formalizar la propiedad de Progreso del \stlcw. El objetivo del
%   trabajo es comprender en profundidad las maneras de formalizar el \stlcw y sus
%   propiedades principales. El asistente a usar será Agda por el mismo motivo,
%   comprender en profundidad esa herramienta en particular.
% \end{abstract}

\section{Avances}

En este mes estuve siguiendo el libro PLFA de Wadler para interiorizarme con el
asistente. Comparto el repositorio en el que estuve copiando las definiciones e
implementando los ejercicios propuestos:
\url{https://github.com/cfsottile/plfa}. Estoy en el capítulo de
Cuantificadores, me quedan los capítulos de Decidibilidad y de Listas, y luego
planeo empezar con la implementación del $\lambda-$cálculo simplemente tipado
para poder probar la propiedad de progreso. No me topé con mayores dificultades
aprendiendo Agda. Sí tuve que detenerme para intentar entender con cierta
profundidad algunos aspectos, como el caso de expresar \texttt{rewrite} en
términos de \texttt{with} (que finalmente no comprendí), la prueba de simetría
de la igualdad de Leibniz que me desacomodó bastante las ideas, y algunos
ejercicios de la parte de negación que no me salieron rápido. En general le
dedico bastante a hacer los ejercicios así entiendo mejor, por eso avanzo lento.
Anoté dos dudas que aprovecho para copiar a continuación.

\newpage

\section{Consultas}

\subsection{Ignorando parámetros sin éxito}

Aclaro que usé este formato inusual de definir las funciones en el
\texttt{where} solo para ver si funcionaba.

\includegraphics[width=\textwidth]{consulta2.png}

\subsection{Normalización de goals pero no de términos}

\includegraphics[width=\textwidth]{consulta1.png}

\end{document}