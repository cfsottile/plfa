\documentclass{article}

\usepackage[T1]{fontenc}
\usepackage{fullpage}
\usepackage{xcolor}
\usepackage[parfill]{parskip}
\usepackage{bm}
\usepackage{xspace}
% \usepackage{amsmath}
% \usepackage{amssymb}

\title{Formalización de la propiedad de Progreso\\para el \stlcw}
% \subtitle{Blueprint}
\author{Cristian Sottile}
\date{14 de septiembre de 2023}

\newcommand{\stlcw}{$\lambda-$cálculo simplemente tipado\xspace}
\newcommand{\stlc}{$\lambda^{\rightarrow}$}
\newcommand{\lamg}{\ensuremath{\lambda^{\mathbb{m}}}\xspace}
\newcommand{\lamm}{\ensuremath{\lambda^{\mathbb{m}}}\xspace}
\newcommand{\wme}{\ensuremath{\mathcal{W}}\xspace}
\newcommand{\wmem}[1]{\ensuremath{\mathcal{W}(#1)}\xspace}
\newcommand{\tmme}{\ensuremath{\mathcal{T}^{\mathbb{m}}}\xspace}
\newcommand{\tmmem}[1][M]{\ensuremath{\mathcal{T}^{\mathbb{m}}(#1)}\xspace}
\newcommand{\tme}{\ensuremath{\mathcal{T}}\xspace}
\newcommand{\tmem}[1][M]{\ensuremath{\mathcal{T}(#1)}\xspace}
\newcommand{\tom}{\ensuremath{\rightarrow_m}\xspace}
\newcommand{\tob}{\ensuremath{\rightarrow_{\beta}}\xspace}
\newcommand{\tof}{\ensuremath{\triangleright}\xspace}
\newcommand{\wrap}[1]{\ensuremath{\bm{\{}#1\bm{\}}}\xspace}
\newcommand{\wei}[1]{\ensuremath{\mathsf{w}(#1)}\xspace}
\newcommand{\maxdeg}[1]{\ensuremath{\dh(#1)}\xspace}
\newcommand{\simp}[1]{\ensuremath{\mathsf{S}_*(#1)}\xspace}
\newcommand{\simpd}[2][d]{\ensuremath{\mathsf{S}_{#1}(#2)}\xspace}

\newcommand{\inte}[1]{\ensuremath{[[#1]]}}
\newcommand{\lam}[2][x]{\ensuremath{\lambda #1 . #2}}

\newcommand{\sep}{\ensuremath{\ |\ }}
\newcommand{\ie}{{\em i.e.}\xspace}
\newcommand{\eg}{{\em e.g.}\xspace}
\newcommand{\ver}[1]{\textcolor{red}{#1}}

\newcommand{\n}[1]{\ensuremath{\mathsf{#1}}}

\begin{document}

\maketitle

\begin{abstract}
  La propuesta es formalizar la propiedad de Progreso del \stlcw. El objetivo del
  trabajo es comprender en profundidad las maneras de formalizar el \stlcw y sus
  propiedades principales. El asistente a usar será Agda por el mismo motivo,
  comprender en profundidad esa herramienta en particular.

\end{abstract}

\section{La propiedad}

En el \stlcw definimos una relación entre
términos que llamamos reducción y expresa la idea de computación. Las formas
normales son los términos que “no reducen”. Los valores son los términos que se
construyen (dependiendo la variante del sistema puede que sea únicamente) con
los constructores de introducción de su tipo. Por ejemplo:

\begin{itemize}
\item el constructor de introducción de las funciones es $\lambda$
\item el constructor de eliminación de las funciones es la aplicación
  (yuxtaposición)
\item el constructor de introducción de los pares es $(\_,\_)$
\item los constructores de eliminación de los pares son las proyecciones
  \n{fst} y \n{snd}
\item los constructores de introducción de los números son \n{zero} y \n{suc}
\end{itemize}

Lo esperable de un lenguaje de programación es que todo término se corresponda
vía reducción con un valor de su tipo. Por ejemplo $(\lambda x.x)\ \n{zero}$ se
corresponde con el valor \n{zero}. Si nuestro lenguaje no contara con la regla
$\beta$, el término $(\lambda x.x)\ \n{zero}$ estaría en forma normal sin
corresponderse con un valor. Llamamos a estos términos “trabados”. La propiedad
de Progreso establece que no existen términos trabados en el \stlcw.

\section{La formalización}

Tenemos dos maneras de expresar progreso en Agda: mediante una definición
inductiva de las pruebas de una proposición, y otra mediante el uso de
conectivos lógicos. Abordaremos ambas, las compararemos y probaremos que son
equivalentes. Previamente deberemos definir varias funciones y lemas referentes
al \stlcw, así como la propia representación del cálculo.

\subsection{Definición inductiva}

Definimos la relación \n{Progress}, que establece cuándo un término progresa,
de la siguiente manera:

\verb|data Progress (M : Term) : Set where|\\
\verb|  step : forall {N} → (M → N) → Progress M|\\
\verb|  done :              Value M → Progress M|

Con el primer constructor indicamos que si hay un término $N$ al que
reduzca $M$, entonces cumple progreso, y con el segundo que si es un valor,
entonces cumple progreso.

\subsection{Conectivos lógicos}

Utilizamos los conectivos lógicos (que previamente definiremos) de disyunción y
cuantificación existencial para enunciar el teorema como una proposición que
establece que siempre que un término tenga tipo bajo el contexto cerrado, ese
término es un valor o bien tiene un reducto.

$$\n{progress} :
  \forall M,A.\
    \emptyset \vdash M : A \rightarrow
      \n{Value}\ M \vee \exists N (M \longrightarrow N)$$

\end{document}